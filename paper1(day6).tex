\documentclass[conference]{IEEEtran}
\IEEEoverridecommandlockouts
% The preceding line is only needed to identify funding in the first footnote. If that is unneeded, please comment it out.
\usepackage{cite}
\usepackage{amsmath,amssymb,amsfonts}
\usepackage{algorithmic}
\usepackage{graphicx}
\usepackage{textcomp}
\usepackage{xcolor}
\def\BibTeX{{\rm B\kern-.05em{\sc i\kern-.025em b}\kern-.08em
    T\kern-.1667em\lower.7ex\hbox{E}\kern-.125emX}}
\begin{document}

\title{Comparison of MLP and CNN in Training on the MNIST Dataset}

\author{\IEEEauthorblockN{Taehyeon Kim}
\IEEEauthorblockA{\textit{Undergraduate Student, Dept. of Computer Engineering} \\
\textit{Kyonggi University}\\
Suwon, Korea \\
dannykim05@kyonggi.ac.kr}
}

\maketitle

\begin{abstract}
This paper compares the performance of MLP and CNN using the MNIST dataset. The experiment shows that CNN performs better than MLP in classifying images. The comparison is based on training loss values to evaluate the effectiveness of each model.

\end{abstract}

\begin{IEEEkeywords}
CNN, MLP, MNIST, Image Classification, Deep Learning
\end{IEEEkeywords}

\section{Introduction}
MLP and CNN are types of artificial neural networks. These models both have nodes, weights, activation functions, etc. However, CNN is more commonly used than MLP in image classification. This paper aims to verify whether CNN truly performs better than MLP in image classification.

\section{Method}
The architectures of MLP and CNN, and training settings such as optimizer, batch size, and epochs are described below.

\textbf{MLP Architecture:}
\begin{itemize}
  \item Flatten layer: reshapes input from [1, 28, 28] to [784]
  \item Fully connected layer: 784 $\rightarrow$ 128
  \item ReLU activation
  \item Fully connected layer: 128 $\rightarrow$ 10
\end{itemize}

\textbf{CNN Architecture:}
\begin{itemize}
  \item 2D Convolutional layer: in channels = 1, out channels = 16, kernel size = 3, padding = 1
  \item Max Pooling layer: kernel size = 2, stride = 2
  \item Flatten layer: reshapes input from [16, 14, 14] to [3136]
  \item Fully connected layer: 3136 $\rightarrow$ 10
\end{itemize}

\textbf{Training Settings:}
\begin{itemize}
  \item Optimizer: SGD
  \item Learning rate: 0.1
  \item Batch size: 64
  \item Epochs: 5
\end{itemize}

\section{Results}

In Epoch 1, loss values are similar. However, by epoch 5, the training loss of MLP fluctuates around $\pm$ 0.1, while CNN’s training loss remains around $\pm$ 0.01. CNN’s loss is ten times smaller than that of MLP.

\begin{table}[htbp]
\caption{MLP Training Loss (Selected Batches)}
\begin{center}
\begin{tabular}{|c|c|c|}
\hline
\textbf{Epoch} & \textbf{Batch Index} & \textbf{Loss} \\
\hline
1 & 0   & 2.2996 \\
1 & 100 & 0.5997 \\
1 & 200 & 0.4387 \\
1 & 300 & 0.4640 \\
1 & 400 & 0.1842 \\
\hline
5 & 600 & 0.0815 \\
5 & 700 & 0.0771 \\
5 & 800 & 0.0886 \\
5 & 900 & 0.1146 \\
\hline
\end{tabular}
\label{tab:training_loss}
\end{center}
\end{table}

\begin{table}[htbp]
\caption{CNN Training Loss (Selected Batches)}
\begin{center}
\begin{tabular}{|c|c|c|}
\hline
\textbf{Epoch} & \textbf{Batch Index} & \textbf{Loss} \\
\hline
1 & 0   & 2.2996 \\
1 & 100 & 0.5997 \\
1 & 200 & 0.4387 \\
1 & 300 & 0.4640 \\
1 & 400 & 0.1842 \\
\hline
5 & 600 & 0.1059 \\
5 & 700 & 0.1204 \\
5 & 800 & 0.0337 \\
5 & 900 & 0.0096 \\
\hline
\end{tabular}
\label{tab:cnn_loss}
\end{center}
\end{table}




\section{Discussion}
CNN performs better than MLP in image classification. One reason is that the MLP model uses a flatten layer at the beginning, which loses local features. In contrast, the CNN model uses convolutional layers that extract local features through kernel operations. Due to this and other factors, CNN achieves lower loss values than MLP.

One limitation of this experiment is that the loss value of MLP slightly increases at the end of epoch 5. Another limitation is that only the loss value was used to compare the two models. Although loss is useful to evaluate performance, it is not sufficient alone. Other factors should also be considered for a more accurate comparison.

\section*{Conclusion}
Both MLP and CNN are artificial neural networks. However, CNN is more widely used in image classification. This experiment suggests that this is because MLP loses spatial information through flattening, while CNN preserves it via convolution. The results demonstrate that the model architecture has a significant impact on performance, and CNN is more suitable for image classification tasks such as MNIST.

\end{document}
